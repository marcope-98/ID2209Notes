\defbibheading{bibempty}{}

\newcommand{\M}[1]{\begin{bmatrix}#1\end{bmatrix}}
\newcommand{\und}[1]{\underline{#1}}
\newcommand{\B}[1]{\begin{Bmatrix}#1\end{Bmatrix}}
\newcommand{\pd}[2]{\cfrac{\partial#1}{\partial#2}}
\newcommand{\td}[2]{\cfrac{d#1}{d#2}}
\newcommand{\tdd}[2]{\cfrac{d^2#1}{d#2^2}}
\newcommand{\pdd}[2]{\cfrac{\partial^2#1}{\partial#2^2}}
\renewcommand{\theta}{\vartheta}
\newcommand{\omissis}{[\textellipsis\unkern]}
\newcommand{\pexp}[2]{\prescript{#1}{}{#2}{}{}}
\newcommand{\ret}[1]{{#1}^{\leftarrow}}
\renewcommand{\epsilon}{\varepsilon}
\newcommand{\smalltodo}[2][]{\todo[caption={#2}, #1, backgroundcolor=white!20!white, bordercolor=white]{\begin{spacing}{0.5}\texttt{#2}\end{spacing}}}
\newcommand{\side}[1]{\smalltodo[size=\footnotesize]{#1}\textbf{#1}}
\newcommand{\pside}[1]{\smalltodo[size=\footnotesize]{#1}}
\DeclarePairedDelimiter\abs{\lvert}{\rvert}
\DeclarePairedDelimiter{\norma}{\lVert}{\rVert}
\newcommand{\degr}[1]{^{\circ\!#1}}
\DeclareMathOperator*{\minimize}{minimize}
\DeclareMathOperator*{\argmin}{argmin}
\DeclareMathOperator*{\argmax}{argmax}
\newtheorem{theorem}{Theorem}

\makeatletter
\newcommand\incircbin
{%
  \mathpalette\@incircbin
}
\newcommand\@incircbin[2]
{%
  \mathbin%
  {%
    \ooalign{\hidewidth$#1#2$\hidewidth\crcr$#1\bigcirc$}%
  }%
}
\newcommand{\ooplus}{\incircbin{+}}
\newcommand{\oominus}{\incircbin{-}}
\newcommand{\oocirca}{\incircbin{\sim}}
\makeatother

\renewcommand{\arraystretch}{1.3}
\setitemize{noitemsep,topsep=10pt,parsep=0pt,partopsep=0pt}
\newcommand{\cmark}{\ding{51}}
\newcommand{\xmark}{\ding{55}}
\newcommand{\cge}{\succeq}
\newcommand{\cle}{\precceq}
\newcommand{\cless}{\precc}
\newcommand{\cmore}{\succ}

\setcounter{minitocdepth}{3}

\definecolor{codegreen}{rgb}{0,0.6,0}
\definecolor{codegray}{rgb}{0.5,0.5,0.5}
\definecolor{codepurple}{rgb}{0.58,0,0.82}
\definecolor{backcolour}{rgb}{0.95,0.95,0.92}

\lstdefinestyle{mystyle}{
    backgroundcolor=\color{backcolour},   
    commentstyle=\color{codegreen},
    keywordstyle=\color{magenta},
    numberstyle=\tiny\color{codegray},
    stringstyle=\color{codepurple},
    basicstyle=\ttfamily\footnotesize,
    breakatwhitespace=false,         
    breaklines=true,                 
    captionpos=b,                    
    keepspaces=true,                 
    numbers=left,                    
    numbersep=5pt,                  
    showspaces=false,                
    showstringspaces=false,
    showtabs=false,                  
    tabsize=2
}

\lstset{style=mystyle}